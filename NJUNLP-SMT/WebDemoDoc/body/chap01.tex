% !TEX TS-program = XeLaTeX
% !TEX encoding = UTF-8 Unicode

\chapter{怎样解析WSDL}
\label{chap01}

下面首先将介绍WSDL及Web Service的概念。然后阐述怎样生成可以使用的WSDL,主要以c\#语言和java语言(\textcolor{red}{TODO})为例介绍。最后以python语言为例,介绍怎样解析生成的WSDL及调用其中的服务。

\section{Wsdl及Web serviceL简介}

在了解WSDL之前,需要首先了解Web Service,Web Service是一种服务导向架构的技术,通过标准的Web协议提供服务,目的是保证不同平台的应用服务可以互操作。

\subsection{Web service介绍}
通俗的说,我们把计算机后台程序提供的功能,称为服务。根据来源的不同,服务可以分为两种,一种是"本地服务"(使用同一台机器提供的服务,不需要网络),另一种是"网络服务"(使用另一台计算机提供的服务,必须通过网络才能完成)。

wikipedia上关于Web Service的描述如下

\url{http://eh.wikipedia.org/wiki/Web_service}

\begin{enumerate}

\item 本地服务的缺点

a) 可移植性差

如果你想把本机的服务,移植到其他机器上,往往很困难,尤其是在跨平台的情况下。

\item Web Service的优势

a) 平台无关

不管使用什么系统,都可以使用Web Service。

b) 编程语言无关

使用规范的格式调用,就可以使用任意编程语言。
\end{enumerate}

\subsection{WSDL简介}

WSDL(Web Services Description Language)指网络服务描述语言。WSDL是使用XML编写的文档,可描述某个Web service,用来规定服务的位置及提供的操作。

wikipedia上关于WSDL的描述如下

\url{http://en.wikipedia.org/wiki/Web_Services_Description_Language}

WSDL文档利用下面的元素来描述某个Web service的

%%插入三线表示例%%
\begin{table}[h]
	\centering
	\begin{tabular}{ccccc}
		\toprule
		 元素& 定义\tabularnewline
		\midrule
		<portType> & web service 执行的操作\tabularnewline
		<message> & web service 使用的消息\tabularnewline
		<types> & web service 使用的数据类型\tabularnewline
          <binding> & web service 使用的通信协议\tabularnewline
		\bottomrule
	\end{tabular}
\end{table}

\section{基本步骤}

\subsection{C\#语言怎样生成可被调用的WSDL}

\begin{enumerate}

\item 如何:在托管Windwos服务中承载WCF服务

参考 \url{http://msdn.microsoft.com/zh-cn/library/ms733069.aspx}

\item 生成WSDL时需要注意的问题

\textcolor{red}{不要直接粘贴配置文件,防止生成不合法字符}

\textcolor{red}{配置文件中的locolhost改为提供服务的服务器IP地址}

\textcolor{red}{修改配置文件中的wsHttpBinding为basicHttpBinding}

\item 修改后的配置文件

\begin{figure}[htpd]
  \centering
  \includegraphics[scale=0.6]{appconfig.jpg}
  \caption{修改后的配置文件}
\end{figure}
\end{enumerate}

\subsection{java语言怎样生成可被调用的WSDL}

\textcolor{red}{TODO}

\subsection{使用python语言解析生成的WSDL}

首先查看需要调用的WSDL文件,这里以提供统计机器翻译的WSDL为例.

地址如下

\url{http://114.212.189.224:9000/NJUPhraZTranslationService?wsdl}

\begin{enumerate}
	
\item SMT WSDL文件

\begin{figure}[htpd]
  \centering
  \includegraphics[scale=0.6]{SMTwsdl.jpg}
  \caption{SMT WSDL 文件}
\end{figure}

\item 使用SUDS解析WSDL

suds的使用说明如下

\url{https://fedorahosted.org/suds/wiki/Documentation}

解析程序如下:

\begin{lstlisting}[language=python]
from suds.client import Client
# The url below links to NJUNLP SMT wsdl and
# This wsdl provides Translate operation
SMTUrl="http://114.212.189.224:9000/NJUPhraZTranslationService?wsdl" 
SMTClient=Client(SMTurl)

# The url below links to NJUNLP WordSegment wsdl and 
# This wsdl provides RunICTCLASSegment operation
WordSegUrl="http://114.212.189.224:9444/WCFICTCLASService?wsdl"
WordSegClient=Client(WordSegUrl)
    
testString=unicode("南京大学","utf-8")

WordSegClient.Client(testString)
print SMTClient.service.Translate(testString)
\end{lstlisting}
\item 需要注意的问题

在使用SMT提供的服务时候,解析服务的文件时需要将文件的编码设置为utf-8.如果设置编码后,还出现服务不能使用的情况,可以进行牵制转换,这里以python语言为例:

\begin{lstlisting}[language=c]
#以南京大学为例
unicode("南京大学","utf-8")
\end{lstlisting}
\end{enumerate}


